\Exercise{Bestehendes erweitern}
%
\par In JavaScript können Sie über die \jvar{prototype} Eigenschaft nicht nur
Ihre Klasse um öffentliche Methoden und Eigenschaften erweitern, sondern auch
bereits bestehende Klassen. In dieser Aufgabe sollen Sie die Array-Klasse um
eine Methode \jfunc{shuffle} erweitern. Diese Methode soll die Elemente des
Arrays zufällig umsortieren.
%
\par Zum Testen Ihrer Implementierung bauen Sie ein Formular das über ein
Textfeld zur Eingabe und zwei Buttons (\emph{Add}, \emph{Shuffle}) verfügt.
Beim Drücken auf \emph{Add} soll die Eingabe einem Array hinzugefügt werden und
das Textfeld anschließend geleert werden. Beim Drücken auf Shuffle soll das
Array über Ihre Methode umsortiert werden und anschließend ausgegeben werden.
Die Ausgabe sollte unterhalb der Eingabe in einem \htag{div}-Container
erfolgen. Dieser Ort würde sich auch eignen um nach jedem Ausführen des
\emph{Add}-Callbacks den aktuellen Inhalt des Arrays anzuzeigen.
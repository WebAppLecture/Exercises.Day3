\RequiredExercise{Die Nullstelle finden}
%
\par Sie erstellen ein sehr einfaches Formular mit zwei \jvar{type=number} ($-\infty$ bis $+\infty$) Elementen und einer \jvar{type=range} (genannt $N$, von $2$ bis $10000$) Eingabe. Es soll eine live, d.h. nicht erst beim Drücken des Submit Buttons, Validierung ausgeführt werden. Hierbei wird überprüft, ob die erste Nummer (genannt $x_i$) kleiner ist als die zweite Nummer (def. $x_f$). Sollte dem nicht so sein, so ist der Submit Button deaktiviert.
%
\par Die verwendete Funktion $f$ soll direkt im JavaScript-Code eingebaut werden. Folgender Code soll implementiert werden:
%
\begin{enumerate}
\item Zunächst berechnen Sie $\delta = (x_f - x_i)/N$.
\item Nun gehen Sie von $x=x_i$ mit $N$ Schritten in Richtung $x_f$.
\item Beim Vorzeichenwechsel, d.h. $f(x_{i - 1}) \cdot f(x_i) < 0$, soll der Punkt $x_i$ einer Liste hinzugefügt werden.
\item Die so gefundenen Punkte sollen am Ende ausgegeben werden (z.B. in einer Textbox).
\end{enumerate}
\RequiredExercise{Zahlenraten}
%
\par Sie entwickeln ein Spiel mit HTML und JavaScript. Gehen Sie folgendermaßen
vor:
%
\begin{itemize}
\item
Basteln Sie eine HTML Seite mit drei \htag{div}-Containern, wobei im ersten
Container drei numerische Boxen (Minimum, Maximum, Zeit in Sekunden) und ein
Button zum Absenden sein sollen.
\item
Im zweiten Container sollen zwei \htag{span} für die verbleibende Zeit, Anzahl der
Versuche, sowie ein \htag{select}-Feld mit Optionen ``Zahl wählen'' und den
verfügbaren Zahlen (von Minimum bis Maximum) angezeigt werden.
\item
Der dritte Container beinhaltet ein Button (``Neustarten'') und ein \htag{span}
zur Anzeige des Ergebnisses.
\item
Schreiben Sie JavaScript in einer externen Datei (\emph{*.js}). Beim Starten
des Scriptes soll eine anonyme Methode selbstständig aufgerufen werden, die
einige Veränderungen am HTML-Code durchführen: Die letzten beiden Container
sollen verborgen werden (entweder mit jQuery über Standardmethoden):
%
\begin{lstlisting}
// jQuery
$('#...'').hide();
// Standard
getElementById('...').style.display = 'none';
\end{lstlisting}
%
\item
Außerdem sollen Ereignisse gesetzt werden. Der Button im ersten Container soll
einen \jvar{onclick}-Callback erhalten, genauso wie der Button im dritten
Container. Das \htag{select}-Feld im zweiten Container soll einen
\jvar{onchange}-Callback erhalten.
\item
Überlegen Sie sich die Callbacks geschickt zu implementieren, so dass folgende
Funktionalität erzeugt wird: Es ist immer nur ein Container sichtbar, und es
ist immer die aktuelle Anzahl an verbleibenden Sekunden bis zum Spielende
sichtbar.
\item
Im 3. Container soll als Zusammenfassung die richtige Zahl (zufällig mit:
%
\begin{lstlisting}
Math.floor(Math.random() * (Maximum - Minimum)) + Minimum
\end{lstlisting}
%
erzeugt), die zur Lösung verstrichene Zeit und die Anzahl der Versuche
angezeigt werden.
\item
Binden Sie Ihre erstellte JavaScript Datei über einen \htag{script} Tag vor dem
Ende des \htag{body} ein.
\end{itemize}